\documentclass[12pt]{article}

% Packages
\usepackage[margin=1in]{geometry}  % Page layout
\usepackage{graphicx}             % For figures
\usepackage{hyperref}             % For clickable links
\usepackage{amsmath}              % For math equations
\usepackage{caption}              % For better captions
\usepackage{listings}             % For code listings
\usepackage{comment}              % For multiline comments

% Title and Author
\title{LaTeX Beginner Tutorial}
\author{Zhihai He}
\date{\today}

\begin{document}

\maketitle
\tableofcontents
\newpage

% 1. Basic Page Layout
\section{Basic Page Layout}
The \texttt{\textbackslash documentclass} command defines the type of document. For example:
\begin{itemize}
    \item \texttt{article}: Suitable for short documents.
    \item \texttt{report}: For longer documents, such as theses.
    \item \texttt{book}: For books.
\end{itemize}

You can adjust the page layout using the \texttt{geometry} package. Example:
\begin{lstlisting}
\usepackage[margin=1in]{geometry}
\end{lstlisting}

% 2. Titles and Sections
\section{Titles and Sections}
Define the title, author, and date of your document:
\begin{lstlisting}
\title{LaTeX Beginner Tutorial}
\author{Zhihai He}
\date{\today}
\maketitle
\end{lstlisting}

To structure your content, use:
\begin{itemize}
    \item \texttt{\textbackslash section\{...\}}: First-level heading.
    \item \texttt{\textbackslash subsection\{...\}}: Second-level heading.
    \item \texttt{\textbackslash subsubsection\{...\}}: Third-level heading.
\end{itemize}

To create unnumbered sections that do not appear in the Table of Contents, use:
\begin{itemize}
    \item \texttt{\textbackslash section*\{...\}}
    \item \texttt{\textbackslash subsection*\{...\}}
    \item \texttt{\textbackslash subsubsection*\{...\}}
\end{itemize}

% 3. New Page, New Line, and Comments
\section{New Page, New Line, and Comments}
\texttt{\textbackslash newpage} starts a new page. \texttt{\textbackslash\textbackslash} creates a new line within the same paragraph. 

\texttt{\%} is used for single-line comments, while multiline comments can be created using the \texttt{comment} package. Examples:

\begin{lstlisting}
% This is a single-line comment.
\end{lstlisting}

\begin{lstlisting}
\usepackage{comment}

\begin{comment}
    This is a multiline comment.
    This is another line in the comment.
\end{comment}
\end{lstlisting}

% 4. Inserting Mathematical Formulas
\section{Inserting Mathematical Formulas}
LaTeX provides multiple ways to insert mathematical formulas, depending on the context and formatting requirements. Below are some commonly used methods:

\subsection*{Inline Formulas}
Inline formulas are embedded within text and written using \texttt{\$...\$} or \texttt{\textbackslash(...\textbackslash)}.

\begin{itemize}
    \item Example using \texttt{\$...\$}: \\
    This is an inline formula: $E = mc^2$.
    \item Example using \texttt{\textbackslash(...\textbackslash)}:\\
    This is another inline formula: \( a^2 + b^2 = c^2 \).
\end{itemize}

\subsection*{Displayed Formulas}
Displayed formulas are centered and occupy their own line. They can be written using \texttt{\textbackslash[\textbackslash]} or the \texttt{equation} environment (with optional numbering).

\begin{itemize}
    \item Without numbering:
    \[
    a^2 + b^2 = c^2
    \]
    \item With numbering:
    \begin{equation}
        E = mc^2
        \label{eq:energy}
    \end{equation}
    The formula \eqref{eq:energy} is Einstein's energy-mass equivalence.
\end{itemize}

\subsection*{Aligned Formulas}
Use the \texttt{align} environment to align multiple equations. The \texttt{\&} symbol specifies the alignment point.

\begin{align}
    a + b &= c \label{eq:line1} \\
    x + y &= z
\end{align}

The equation \eqref{eq:line1} shows an example of aligned formulas.

\subsection*{Cases and Matrices}
For piecewise functions or matrices, special environments like \texttt{cases} and \texttt{bmatrix} are used.

\begin{itemize}
    \item Piecewise function using \texttt{cases}:
    \[
    f(x) =
    \begin{cases} 
        x^2 & \text{if } x \geq 0, \\
        -x  & \text{if } x < 0.
    \end{cases}
    \]
    \item Matrix using \texttt{bmatrix}:
    \[
    A=
    \begin{bmatrix}
        1 & 2 \\
        3 & 4
    \end{bmatrix}
    \]
\end{itemize}

\subsection*{Custom Formula Tags}
You can customize the numbering of equations using the \texttt{\textbackslash tag} command.

\[
a^2 + b^2 = c^2 \tag{Pythagoras}
\]

\subsection*{Unnumbered Formulas}
To suppress numbering for equations, use the \texttt{align*} or \texttt{equation*} environments.

\begin{align*}
    a + b &= c \\
    x + y &= z
\end{align*}

\subsection*{Some Mathematical Symbols}
LaTeX provides a rich set of mathematical symbols for use in equations. Examples include:

\begin{itemize}
    \item Operators: \( +, -, \times, \div \)
    \item Greek letters: \( \alpha, \beta, \gamma, \pi \)
    \item Logical symbols: \( \forall, \exists, \neg, \wedge, \vee \)
    \item Set symbols: \( \cup, \cap, \subseteq, \infty \)
    \item Arrows: \( \rightarrow, \Leftarrow, \iff \)
\end{itemize}

% 5. Adding Figures
\section{Adding Figures}
To include an image, use the \texttt{graphicx} package. Example:
\begin{lstlisting}
\usepackage{graphicx}

\begin{figure}[h]
    \centering
    \includegraphics[width=0.5\textwidth]{images/example.jpg}
    \caption{Sample Image}
    \label{fig:sample}
\end{figure}
\end{lstlisting}

Ensure the image is in the \texttt{images/} folder.

% 6. Creating Tables
\section{Creating Tables}
Tables can be created using the \texttt{tabular} environment. Example:
\begin{lstlisting}
\begin{table}[h]
    \centering
    \begin{tabular}{|c|c|c|}
        \hline
        Column 1 & Column 2 & Column 3 \\
        \hline
        Data 1   & Data 2   & Data 3   \\
        \hline
    \end{tabular}
    \caption{Sample Table}
    \label{tab:sample}
\end{table}
\end{lstlisting}

% 7. Adding Footnotes
\section{Adding Footnotes}
Add footnotes using \texttt{\textbackslash footnote\{...\}}. Example:
\begin{lstlisting}
This is a sample footnote.\footnote{Here is the footnote content.}
\end{lstlisting}
This a sampe footnote\footnote{Here is the footnote content}.
% 8. Using BibTeX for References
\section{Using BibTeX for References}
To manage references, create a \texttt{.bib} file (e.g., \texttt{references.bib}) with the following content:
\begin{lstlisting}
@article{vaswani2017attention,
  title={Attention is all you need},
  author={Vaswani, A},
  journal={Advances in Neural Information Processing Systems},
  year={2017}
}
\end{lstlisting}

In your LaTeX file, use:
\begin{lstlisting}
\bibliographystyle{plain}
\bibliography{references}
\end{lstlisting}
Attenion is all you need\cite{vaswani2017attention}.

% 9. Custom: LaTeX Programming Practices
\section{Custom: LaTeX Programming Practices}
\label{sec:customs}
When writing in LaTeX, it's important to follow good programming practices to ensure the document is well-structured, easy to maintain, and visually appealing. Below are some commonly recommended LaTeX coding habits:

\subsection*{Use Logical Sections}
Organize your document into sections, subsections, and subsubsections using:
\begin{itemize}
    \item \texttt{\textbackslash section\{...\}}: For major sections.
    \item \texttt{\textbackslash subsection\{...\}}: For subsections.
    \item \texttt{\textbackslash subsubsection\{...\}}: For detailed sections.
\end{itemize}
Keep sections concise and avoid overusing \texttt{\textbackslash subsubsection} unless necessary.

\subsection*{Break Lines Thoughtfully}
LaTeX ignores single line breaks in the source file. To indicate a new paragraph, leave an empty line between two paragraphs. For example:
\begin{lstlisting}
This is the first paragraph.

This is the second paragraph.
\end{lstlisting}
Do not use \texttt{\textbackslash\textbackslash} for creating new paragraphs. Instead, reserve it for line breaks within the same paragraph.

\subsection*{Use Comments for Clarity}
Use \texttt{\%} to add comments to your LaTeX file. Comments are ignored during compilation and can be helpful for:
\begin{itemize}
    \item Explaining complex parts of the code.
    \item Temporarily disabling sections during debugging.
\end{itemize}

Example:
\begin{lstlisting}
% This is a single-line comment
\end{lstlisting}

For longer comments, consider using the \texttt{comment} package:
\begin{lstlisting}
\usepackage{comment}

\begin{comment}
This section is temporarily disabled.
\end{comment}
\end{lstlisting}

\subsection*{Keep Lines Short}
Limit the length of each line in your LaTeX source to around 80 characters. This makes the file easier to read and edit, especially in version control systems.

\subsection*{Use Meaningful Labels}
When labeling sections, figures, and tables, use clear and consistent names. For example:
\begin{lstlisting}
\label{sec:introduction} % For sections
\label{fig:sample}       % For figures
\label{tab:results}      % For tables
\end{lstlisting}
Example:\ref{sec:customs} is about LaTeX programming custom.
This makes cross-referencing easier and improves the readability of your code.

\subsection*{Avoid Hardcoding Layout}
Instead of manually adjusting spaces or positions, use LaTeX commands and environments designed for layout. For example:
\begin{itemize}
    \item Use \texttt{\textbackslash vspace} and \texttt{\textbackslash hspace} sparingly.
    \item Avoid fixed-width adjustments like \texttt{\textbackslash hskip}.
    \item Use packages like \texttt{geometry} and \texttt{graphicx} for flexible layout control.
\end{itemize}

\subsection*{Test and Compile Frequently}
Frequent compilation helps catch errors early. If using Overleaf, take advantage of the real-time preview feature. For large projects, compile a small subset of the document during debugging.

By following these practices, you can ensure your LaTeX document is easy to maintain and produces high-quality output.

% 10. Conclusion
\section{Conclusion}
This tutorial introduced the basics of LaTeX, including:
\begin{itemize}
    \item Setting up the page layout.
    \item Inserting Mathematical Formulas
    \item Adding titles, figures, tables, and footnotes.
    \item Custom
    \item Managing references using BibTeX.
\end{itemize}

For more information, visit \url{https://www.overleaf.com/learn/latex/Learn_LaTeX_in_30_minutes}.

\bibliographystyle{plain}
\bibliography{papers}

\end{document}
